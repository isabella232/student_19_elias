%%%%%%%%%%%%%%%%%%%%%%%%%%%%%%%%%%%%%%%%%%%%%%%%%%%%%%%%%%%
% EPFL report package, main thesis file
% Goal: provide formatting for theses and project reports
% Author: Mathias Payer <mathias.payer@epfl.ch>
%%%%%%%%%%%%%%%%%%%%%%%%%%%%%%%%%%%%%%%%%%%%%%%%%%%%%%%%%%%
\documentclass[a4paper,11pt,oneside]{report}
% Options: MScThesis, BScThesis, MScProject, BScProject
\usepackage[MScProject,lablogo]{EPFLreport}
\usepackage{xspace}

\title{ONet Implementation of Gossip-based Signature Aggregation}
\author{Elias Manuel Poroma Wiri}
\responsible{Prof. Bryan Ford}
\supervisor{Gaylor Bosson}
\othersupervisor{Cristina Basescu}
%\coadviser{Second Adviser}
%\expert{The External Reviewer}

\newcommand{\sysname}{FooSystem\xspace}

\begin{document}
\maketitle
%\makededication
%\makeacks

\begin{abstract}
The \sysname tool enables lateral decomposition of a multi-dimensional
flux compensator along the timing and space axes.

The abstract serves as an executive summary of your project.
Your abstract should cover at least the following topics, 1-2 sentences for
each: what area you are in, the problem you focus on, why existing work is
insufficient, what the high-level intuition of your work is, maybe a neat
design or implementation decision, and key results of your evaluation.

TENTATIVE TOTAL SIZE 19 pages without appendix
\end{abstract}

\maketoc

%%%%%%%%%%%%%%%%%%%%%%
\chapter{Introduction}
%%%%%%%%%%%%%%%%%%%%%%

% The introduction is a longer writeup that gently eases the reader into your
% thesis~\cite{dinesh20oakland}. Use the first paragraph to discuss the setting.
% In the second paragraph you can introduce the main challenge that you see.
% The third paragraph lists why related work is insufficient.
% The fourth and fifth paragraphs discuss your approach and why it is needed.
% The sixth paragraph will introduce your thesis statement. Think how you can
% distill the essence of your thesis into a single sentence.
% The seventh paragraph will highlight some of your results
% The eights paragraph discusses your core contribution.

Introduction to the topic, describe challenge, describe content of chapters.

Tentative size: 1 page.

%%%%%%%%%%%%%%%%%%%%
\chapter{Background}
%%%%%%%%%%%%%%%%%%%%

% The background section introduces the necessary background to understand your
% work. This is not necessarily related work but technologies and dependencies
% that must be resolved to understand your design and implementation.

Tentative size: 4 pages.

\section{Gossip protocols}
Theory about gossip protocols.

\section{Multi-signatures}
Theory about multi-signatures.

\section{Existing protocols using Cothority}
Definition of Cothority and introduction to existing protocols.

\subsection{BLS CoSi}
Description of existing BLS CoSi protocol.

\subsection{Gossip protocol with binary tree aggregation}
Description of existing gossip protocol with binary tree signature aggregation.

\section{The Cothority Overlay Network Library - ONet}
Definition of ONet.


%%%%%%%%%%%%%%%%%%%%%%%%
\chapter{Design and Implementation}
%%%%%%%%%%%%%%%%%%%%%%%%

% Introduce and discuss the design decisions that you made during this project.
% Highlight why individual decisions are important and/or necessary. Discuss
% how the design fits together.
% 
% The implementation covers some of the implementation details of your project.
% This is not intended to be a low level description of every line of code that
% you wrote but covers the implementation aspects of the projects.

Tentative size: 4 pages.

\section{Gossip protocols}

\subsection{Mask protocol}
Description of message content and how signature aggregation is done. 

\subsection{Mask aggregation protocol}
Description of message content and how signature aggregation is done. 

\subsection{Aggregation and Subtraction protocol}
Description of message content and how signature aggregation is done. 

\section{ONet mixed protocol}
Description of message content and how rumor propagation is done in ONet.
Description of gossip protocol implemented using ONet rumor propagation. 

%%%%%%%%%%%%%%%%%%%%
\chapter{Evaluation and Results}
%%%%%%%%%%%%%%%%%%%%

% In the evaluation you convince the reader that your design works as intended.
% Describe the evaluation setup, the designed experiments, and how the
% experiments showcase the individual points you want to prove.

Tentative size: 2 pages.

\section{Parameters used in simulations}
Description of environment parameters used in simulations.

\section{Comparison of new gossip protocols and BLS CoSi}
Show figures/charts with small description.

\section{Comparison of mixed gossip in ONet and BLS CoSi}
Show figures/charts with small description.

%%%%%%%%%%%%%%%%%%%%%%
\chapter{Analysis}
%%%%%%%%%%%%%%%%%%%%%%

Tentative size: 2 pages.

\section{Limitations}
Describe the limitations of the implementation done.

\section{Future Work}
Describe possible improvements.

%%%%%%%%%%%%%%%%%%%%
\chapter{Conclusion}
%%%%%%%%%%%%%%%%%%%%

In the conclusion you repeat the main result and finalize the discussion of
your project. Mention the core results and why as well as how your system
advances the status quo.

Tentative size: 1 page.

\cleardoublepage
\phantomsection
\addcontentsline{toc}{chapter}{Bibliography}
\printbibliography

% Appendices are optional
\appendix
%%%%%%%%%%%%%%%%%%%%%%%%%%%%%%%%%%%%%%
\chapter{Appendix: Install and run instructions}
%%%%%%%%%%%%%%%%%%%%%%%%%%%%%%%%%%%%%%

To install do this and this.

To run do this and this.

Tentative size: 1 page.

%%%%%%%%%%%%%%%%%%%%%%%%%%%%%%%%%%%%%%
\chapter{Appendix: Full simulation results}
%%%%%%%%%%%%%%%%%%%%%%%%%%%%%%%%%%%%%%

The simulation results that were not used in the main body of the report, this
will contain figures/charts generated using the simulation results.

Tentative size: 10 pages.



% You need the following items:
% \begin{itemize}
% \item A box
% \item Crayons
% \item A self-aware 5-year old
% \end{itemize}

\end{document}